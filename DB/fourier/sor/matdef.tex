% ez valahol kellett
\DeclarePairedDelimiter\ceil{\lceil}{\rceil}
\DeclarePairedDelimiter\floor{\lfloor}{\rfloor}

% zárójelek stb.
\newcommand{\Gzjel}[1]{%
{ \left( #1 \right) }
}

\newcommand{\gzjel}[1]{%
{ \left( #1 \right) }
}

\newcommand{\Toligv}[2]{%
#1,\hdots ,#2
}

\newcommand{\GZJ}[1]{%
{ \left( #1 \right) }
}

\newcommand{\KZJ}[1]{%
{ \{ #1 \} }
}

\newcommand{\szzjel}[1]{%
{ \left[ #1 \right] }
}

\newcommand{\Tolig}[2]{%
#1\hdots #2
}

\newcommand{\SZOR}[2]{%
#1\cdot\hdots\cdot #2
}

% integrál rendes d-vel
\newcommand{\mint}[2]{%
\int #1 \text{d}#2
}

% integrál rendes d-vel
\newcommand{\mhint}[4]{%
\int_{#1}^{#2} #3 \text{d}#4
}

\newcommand{\mhpfv}[3]{%
\left [ #3 \right ]_{ #1 }^{ #2 }
}

\newcommand{\mder}[2]{%
\frac{\text{d} #1}{\text{d}#2}
}


% ez máshogy van alapban
\newcommand{\tg}[0]{%
\tan
}

\newcommand{\ctg}[0]{%
\cot
}

\newcommand{\D}[1]{%
{{\mathbb{D}\szzjel{#1}}}
}

\newcommand{\V}[1]{%
{{{\mathbb{D}}^{2}\szzjel{#1}}}
}


\newcommand{\E}[1]{%
{{\mathbb{E}\szzjel{#1}}}
}

\newcommand{\A}[1]{%
{\mathbb{A}\szzjel{#1}}
}

\newcommand{\hehu}[1]{%
\hspace{0.5cm}#1\hspace{0.5cm}
}

\newcommand{\bmx}[1]{%
\begin{bmatrix}
#1
\end{bmatrix}
}

\newcommand{\mx}[1]{%
\begin{matrix}
#1
\end{matrix}
}

\newcommand{\gjmx}[2]{%
\left[
\begin{array}{@{}c|c@{}}
#1 & #2
\end{array}
\right]
}



