Tegyük fel, hogy az $f(x)=\sin(\frac{100}{x})$ függvény értékei $h=0.001$ lépésközzel
adottak a $[0.5, 2\pi ]$ intervallumon. Deriváljuk numerikusan a függvényt! Ábrázoljuk
az eredményt a függvény tényleges deriváltjával közös ábrán. Magyarázzuk meg az eltérést
a $\frac{\sin(3x)}{x}$ fv-nél látottakhoz képest.


