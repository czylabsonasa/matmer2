Egy test \mcode{10} perc alatt \mcode{100} C fokról \mcode{60} C fokra 
hűlt le. A környező levegő hőmérsékletét konstans \mcode{20} C foknak 
tekinthetjük. Mikor hűl le a test \mcode{25} C fokra, ha a test 
hűlésének sebessége egyenesen arányos a test és az őt körülvevő 
levegő hőmérsékletének különbségével? 
(bővebben: \href{https://en.wikipedia.org/wiki/Newton%27s_law_of_cooling}{Newton law of cooling})