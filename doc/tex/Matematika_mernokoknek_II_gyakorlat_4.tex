\documentclass[table]{beamer} 
\mode<presentation>
{
  %\setbeamertemplate{background canvas}[vertical
  %shading][bottom=red!10,top=blue!10]  

 \usetheme{Boadilla}
  \usefonttheme[onlysmall]{structurebold}
}
\usepackage{xcolor}
\usepackage{amsmath,amssymb,latexsym,epsfig,bbm,epic,eepic,multicol,enumerate}
\usepackage{verbatimbox,upquote,fancyvrb}
\usepackage{epstopdf}
\usepackage[utf8]{inputenc}

\author[Baran \'Agnes, Burai P\'al, Nosz\'aly Csaba]{Baran \'Agnes, Burai P\'al, Nosz\'aly Csaba}
\title{Matematika M\'ern\"ok\"oknek 2.}
\date[Gyakorlat]{Gyakorlat\\Fourier transzformáció}

%-----------------------------------------------------%------------------------------------------------------
%-------------------------------------------------------
%------------------------------------------------------
\begin{document} 
\begin{frame}
\maketitle
\end{frame}
%------------------------------------------------------
%------------------------------------------------------ 

\begin{frame}
{Fourier transzformáció}

\begin{exampleblock}{P\'elda}
Határozzuk meg a következő függvény Fourier transzformáltját!
\[
f(x)=\begin{cases}
1,&\mbox{ha }x\in\left[-\frac 12,\frac 12\right]\\
0,&\mbox{egyébként.}
\end{cases}
\]
\end{exampleblock}
{\bf Megold\'as:} Tudjuk, hogy 
\[
{\cal F}[f](\omega) = \int\limits_{-\infty}^{\infty}{f(x)e^{-i\omega x}}{dx}\quad \text{\'es}\quad 
e^{ix}=\cos(x)+i\sin(x), 
\]
tov\'abb\'a a $\cos$ p\'aros, a $\sin$ p\'aratlan f\"uggv\'eny, \'igy 
\[
e^{-ix}=\cos(x)-i\sin(x).
\]

\end{frame}

%------------------------------------------------------
%------------------------------------------------------ 


\begin{frame}
Ebb\H ol  
\[
\int\limits_{-\infty}^{\infty}{f(t)e^{-i\omega x}}{dx}=\int\limits_{-1/2}^{1/2} \cos(\omega x)dx 
-i\underbrace{\int\limits_{-1/2}^{1/2} \sin(\omega x)dx}_{=0}
\]
\[
=2\int\limits_{0}^{1/2} \cos(\omega x)dx =2\left[ \frac{\sin(\omega x)}{\omega}\right] _{x=0}^{\frac 12}
=\frac {2}{\omega}\sin\left( \frac{\omega}{2}\right)
\]
\vskip .5cm 
Teh\'at 
\[
{\cal F}[f](\omega)=\frac {2}{\omega}\sin\left( \frac{\omega}{2}\right)
\]
\end{frame}

%------------------------------------------------------
%------------------------------------------------------ 


\begin{frame}[fragile]
{Fourier transzformáció}

\begin{exampleblock}{1. feladat}
Határozzuk meg a következő függvények Fourier transzformáltját!
\begin{enumerate}
\item $f(x)=\begin{cases}
x,&\mbox{ha }x\in[0,1]\\
0,&\mbox{egyébként.}
\end{cases}$
\item $f(x)=\begin{cases}
|x|,&\mbox{ha }x\in[-1,1]\\
0,&\mbox{egyébként.}
\end{cases}$
\item $f(x)=\begin{cases}
e^{-x},&\mbox{ha }x>0\\
0,&\mbox{egyébként.}
\end{cases}$
\item $f(x)=e^{-|x|}$,
\item $f(x)=\begin{cases}
1+x,&\mbox{ha }x\in[-1,0]\\
1-x,&\mbox{ha }x\in[0,1]\\
0,&\mbox{egyébként.}
\end{cases}$
\end{enumerate}
\end{exampleblock}
\end{frame}
%------------------------------------------------------
%------------------------------------------------------

\begin{frame}[fragile]
{Fourier transzformáció}
\begin{exampleblock}{P\'elda.}
Határozzuk meg a következő függvény Fourier transzformáltját!
\[
f(x)=\begin{cases}
1,&\mbox{ha }x\in[2,3]\\
0,&\mbox{egyébként.}
\end{cases}
\]
\end{exampleblock}
{\bf Megold\'as.} Sz\'amolhatunk a defin\'ici\'o alapj\'an, de haszn\'alhatjuk az al\'abbi 
\"osszef\"ugg\'est is: 

ha az $f$ Fourier-transzform\'altja $F$, akkor 
\[
{\cal F}[f(x-x_0)](\omega)=e^{-i\omega x_0}F(\omega). 
\]
Tudjuk, hogy ha 
\[
g(x)=\begin{cases}
1,&\mbox{ha }x\in\left[-\frac 12,\frac 12\right]\\
0,&\mbox{egyébként,}
\end{cases}
\]
akkor 
\[
{\cal F}[g](\omega)=\frac {2}{\omega}\sin\left( \frac{\omega}{2}\right).
\]
\end{frame}
%------------------------------------------------------
%------------------------------------------------------


%------------------------------------------------------
%------------------------------------------------------

\begin{frame}
Mivel $f(x)=g\left( x-\frac 52\right)$, ez\'ert 
\begin{align*}
{\cal F}[f](\omega)&={\cal F}\left[ g\left( x-\frac 52\right)\right](\omega) =
e^{ -\frac 52 i\omega} \frac {2}{\omega}\sin\left( \frac{\omega}{2}\right)\\
&=\left( \cos\frac{5\omega}{2}-i\sin\frac{5\omega}{2}\right) 
\frac {2}{\omega}\sin\left( \frac{\omega}{2}\right).
\end{align*}

\vskip .5cm
\begin{exampleblock}{2. feladat}
Határozzuk meg a következő függvény Fourier transzformáltját!
\[
f(x)=\begin{cases}
-1,&\mbox{ha }x\in[-1,0]\\
1,&\mbox{ha }x\in]0,1]\\
0,&\mbox{egyébként.}
\end{cases}
\]
\end{exampleblock}
\end{frame}
%------------------------------------------------------
%------------------------------------------------------

\begin{frame}[fragile]
{Fourier transzformáció}
\begin{exampleblock}{P\'elda.}
Ha tudjuk, hogy az $f(x)=e^{-|x|}$ f\"uggv\'eny Fourier transzform\'altja 
${\cal F}[f](\omega)=\frac{2}{1+\omega^2}$, akkor hat\'arozzuk meg a $g(x)=e^{-|2x+3|}$ 
Fourier transzform\'altj\'at. 
\end{exampleblock}
{\bf Megold\'as.} Ha az $f$ Fourier-transzform\'altja $F$ \'es $a\ne 0$, akkor 
\[
{\cal F}[f(x-x_0)](\omega)=e^{-i\omega x_0}F(\omega) \text{ \'es }
{\cal F}\left[ f\left(\frac xa\right)\right] (\omega)=|a|{\cal F}[f](a\omega). 
\]
Most $g(x)=h(2x)$, ahol $h(x)=e^{-|x+3|}=f(x+3)$, \'igy 
\begin{align*}
{\cal F}[g](\omega)&={\cal F}[h(2x)](\omega)=\frac 12{\cal F}[h]\left(\frac{\omega}{2}\right)
=\frac 12{\cal F}[f(x+3)]\left(\frac{\omega}{2}\right)\\
&=\frac 12 e^{\frac 32 i\omega}{\cal F}[f]\left(\frac{\omega}{2}\right)=\frac 12 e^{\frac 32 i\omega}
\frac{2}{1+\left(\frac{\omega}{2}\right)^2}=e^{\frac 32 i\omega}\frac{4}{4+\omega^2}.
\end{align*}
\end{frame}


%------------------------------------------------------
%------------------------------------------------------
\begin{frame}
{Fourier transzformáció}
\begin{exampleblock}{3. feladat}
Hat\'arozza meg az al\'abbi f\"uggv\'enyek Fourier transzform\'altj\'at. 
\begin{multicols}{2}
\begin{enumerate}
\item $g(x)=e^{-|1-x|}$ 
\item $g(x)=5e^{-\left\vert \frac x3-2\right\vert }$ 
\item $g(x)=2e^{-|3x-1|}-e^{-|x+2|}$ 
\end{enumerate}
\end{multicols}
\end{exampleblock}



\begin{exampleblock}{4. feladat}
Hat\'arozza meg a $g$ f\"uggv\'eny Fourier transzform\'altj\'at, ha tudjuk, hogy az 
\[
f(x)=\begin{cases}
e^{-x},&\mbox{ha }x>0\\
0,&\mbox{egyébként.}
\end{cases}
\]
f\"uggv\'eny Fourier transzform\'altja $F(\omega )=\frac{1}{1+\omega i}$.
\[
g(x)=\begin{cases}
e^{4x},&\mbox{ha }x<0\\
0,&\mbox{egyébként.}
\end{cases}
\]
\end{exampleblock}
\end{frame}

%------------------------------------------------------
%------------------------------------------------------

\begin{frame}[fragile]
{Fourier transzformáció}
\begin{exampleblock}{P\'elda.}
Hat\'arozzuk meg a $g$ f\"uggv\'eny Fourier transzform\'altj\'at, ha tudjuk, hogy az $f$ 
Fourier transzform\'altja $F(\omega )=\frac{2}{\omega^2}(1-\cos(\omega))$. \'Abr\'azoljuk 
k\"oz\"os \'abr\'an az $f$ \'es $g$ f\"uggv\'enyeket, egy m\'asik \'abr\'an a Fourier transzform\'altakat. 
\[
f(x)=\begin{cases}
1+x,&\mbox{ha }x\in[-1,0]\\
1-x,&\mbox{ha }x\in[0,1]\\
0,&\mbox{egyébként.}
\end{cases}
\]
\[
g(x)=\begin{cases}
1+\frac x2,&\mbox{ha }x\in[-2,0]\\
1-\frac x2,&\mbox{ha }x\in[0,2]\\
0,&\mbox{egyébként.}
\end{cases}
\]
\end{exampleblock}
{\bf Megold\'as.}
\[
{\cal F}[g](\omega)={\cal F}\left[ f\left(\frac x2\right)\right](\omega)
=2{\cal F}[f](2\omega)=\frac{1}{\omega ^2}(1-\cos(2\omega)).
\]
\end{frame}


%------------------------------------------------------
%------------------------------------------------------

\begin{frame}
\begin{center}
\includegraphics[scale=0.6]{fourier_fr_spekt.eps}
\end{center}

\vskip .5cm 
{\color{red}Lassan v\'altoz\'o jel $\to$ keskeny frekvenciaspektrum}

{\color{blue}Gyorsan v\'altoz\'o jel $\to$ sz\'eles frekvenciaspektrum}
\end{frame}


%------------------------------------------------------
%------------------------------------------------------




%------------------------------------------------------
%------------------------------------------------------

\begin{frame}[fragile]
{Fourier transzformáció}
\begin{exampleblock}{5. feladat}
Legyen $a,b>0,\ x_0\in\mathbb{R}$ adott és $f$ az $x_0-\tfrac{b}{2}$ ponttól az $x_0+\tfrac{b}{2}$ 
pontig terjedő $a$ magasságú téglalap alakú impulzus. Határozzuk meg a Fourier transzformáltját a 
paraméterek függvényében! Hogyan változik a transzformált, ha változtatjuk a paramétereket?
\end{exampleblock}
\begin{exampleblock}{6. feladat}
Tudjuk, hogy $\mathcal{F}\left[e^{-\frac{x^2}{2}}\right]=\sqrt{2\pi}e^{-\frac{\omega^2}{2}}$.
Mi lesz a Fourier transzformáltja a következő függvényeknek?
\begin{itemize}
\item $f(x)=3e^{-2(x-3)^2}$,
\item $g(x)=e^{-x^2+2x}$,
\end{itemize}
Minek a Fourier transzformáltja $F(\omega)=3e^{-2(\omega-3)^2}$ ?
\end{exampleblock}
\end{frame}
%------------------------------------------------------
%------------------------------------------------------
\begin{frame}[fragile]
{Fourier transzformáció}
\begin{exampleblock}{Példa}
Számítsuk ki Matlab segítségével az $f(x)=e^{-|x|}$ Fourier transzformáltját, majd ábrázoljuk a függvényt és a transzformáltat.
\end{exampleblock}
\textbf{Megoldás:}
A \texttt{fourier} parancs segítségével számítsuk ki a Fourier transzformáltat!
\begin{verbatim}
>> syms x
>> f = exp(-abs(x));
>> F = fourier(f)
F =
 2/(w^2 + 1)
\end{verbatim}

%\vskip .5 cm 
L\'athatjuk, hogy a kor\'abban kisz\'amolt 
\[
F(\omega)=\frac{2}{1+\omega ^2}
\]
f\"uggv\'enyt kaptuk. 
\end{frame}

%------------------------------------------------------
%------------------------------------------------------
\begin{frame}[fragile]
Ábrázoljuk a függvényt és a transzformáltat!


\begin{verbatim}
figure;
set(gcf,'units','points','position',[20,20,600,200])
subplot(1,2,1)
fplot(f,'LineWidth',2); axis([-4,4,-.5,2.5]);
title('Az f függvény');
ax=gca; 
ax.XAxisLocation = 'origin';
ax.YAxisLocation = 'origin';
subplot(1,2,2)
fplot(F,'LineWidth',2); axis([-4,4,-.5,2.5]);
title('Az f Fourier transzformáltja');
ax=gca; 
ax.XAxisLocation = 'origin';
ax.YAxisLocation = 'origin';
\end{verbatim}
\end{frame}
%------------------------------------------------------
%------------------------------------------------------
\begin{frame}[fragile]
{Fourier transzformáció}
\begin{center}
\includegraphics[scale=0.6]{fourier_tr1.eps}
\end{center}
\end{frame}
%------------------------------------------------------
%------------------------------------------------------
\begin{frame}[fragile]
{Fourier transzformáció}
\begin{exampleblock}{7. feladat}
Határozzuk meg Matlab segítségével az 1. feladatban szereplő függvények Fourier transzformáltját! 
Ábrázoljuk a függvényt, illetve egy m\'asik \'abr\'an a transzformált valós \'es k\'epzetes r\'esz\'et!
\end{exampleblock}
\end{frame}
%------------------------------------------------------
%------------------------------------------------------
\begin{frame}[fragile]
{Fourier transzformáció}
\begin{exampleblock}{Példa}
Számítsuk ki Matlab segítségével az $F(x)=e^{-x^2}$ inverz Fourier transzformáltját, majd ábrázoljuk a függvényt és a transzformáltat egy ábrán.
\end{exampleblock}
\textbf{Megoldás:}
Az \texttt{ifourier} parancs segítségével számítsuk ki az inverz Fourier transzformáltat!
\begin{verbatim}
>> syms t x
>> F = exp(-(x)^2);
>> f = ifourier(F,x,t)
\end{verbatim}
Ábrázoljuk a függvényt és a transzformáltat közös ábrán 
közös ábrán!
\begin{verbatim}
>> fplot([f F])
>> legend('f','F');
>> axis([-4,4,-.5,1.5])
>> ax = gca;
>> ax.XAxisLocation = 'origin'; ax.YAxisLocation = 'origin';
\end{verbatim}
\end{frame}
%------------------------------------------------------
%------------------------------------------------------
\begin{frame}[fragile]
{Fourier transzformáció}
\includegraphics[scale=0.8]{inverz_fourier_transzformalt1.eps}
\end{frame}

%------------------------------------------------------
%------------------------------------------------------
\begin{frame}[fragile]
{Diszkrét Fourier transzformáció}
\begin{block}
{Példa}
{Határozzuk meg az $x=[2,3, -1, 1]$ $4$-pontú jel Diszkrét Fourier transzformáltját kézzel!
 }
\end{block}
\textbf{Megoldás:}
\[
F_k=\sum\limits_{n=0}^{N-1}f_ne^{-i\frac{2\pi}{N}kn},\qquad k=0,1,\dots,N-1,
\] 
azaz
\[
F_k=\sum\limits_{n=0}^{3}f_ne^{-i\frac{2\pi}{4}kn},\qquad k=0,1,2,3.
\] 
Így
\[
F_0= 2+3-1+1=5,\quad F_1=2+3e^{-i\frac{2\pi}{4}}-e^{-i\frac{2\pi 2}{4}}+e^{-i\frac{2\pi 3}{4}}=3-2i,
\]
\[
F_2=2+3e^{-i\frac{2\pi 2}{4}}-e^{-i\frac{2\pi 4}{4}}+e^{-i\frac{2\pi 6}{4}}=-3,\quad F_4=\cdots=3+2i.
\]
\end{frame}

%------------------------------------------------------
%------------------------------------------------------
\begin{frame}[fragile]
{Diszkrét Fourier transzformáció ($DFT$)}

\begin{exampleblock}{1. feladat}
Határozzuk meg a következő vektorok $DFT$-jét! Írjuk fel a megoldást mátrixos alakban is!
\begin{itemize}
\item $x = [20, 5]$;
\item $x = [3, 2, 5, 1]$.
\end{itemize}
\end{exampleblock}

\begin{exampleblock}{2. feladat}
\begin{itemize}
\item Egy $9$-pontú jel $DFT$-jének páros koordinátái a következők
$F_0=3.1,\ F_2=2.5+4.6i,\ F_4=-1.7+5.2i,\ F_6=9.3+6.3i,\ F_8=5.5-8i$. 
Határozza meg a hiányzó koordinátákat!
\item Pótolja a következő $DFT$ hiányzó adatait:
$F=\left[ 1 - 0i, ?, 3 + 1i, 4 - 1i, 1 - 0i, ?, ?, 2 - 2i \right]$
\end{itemize}
\end{exampleblock}
\end{frame}
%------------------------------------------------------
%------------------------------------------------------
\begin{frame}[fragile]
{Inverz Diszkrét Fourier transzformáció}
\begin{block}
{Példa}
{Határozzuk meg az $x=[5, 3-2i, -3, 3+2i]$ $4$-pontú jel Inverz Diszkrét Fourier transzformáltját kézzel!
 }
\end{block}
\textbf{Megoldás:}
\[
f_k=\frac1N\sum\limits_{n=0}^{N-1}F_ne^{i\frac{2\pi}{N}kn},\qquad k=0,1,\dots,N-1,
\] 
azaz
\[
f_k=\frac14\sum\limits_{n=0}^{3}F_ne^{i\frac{2\pi}{4}kn},\qquad k=0,1,2,3.
\] 
Így
{\small
\[
f_0=\frac{1}{4}(5+ 3-2i -3+3+2i)=2,\quad f_1=\frac{1}{4}(5+(3-2i)e^{i\frac{2\pi}{4}} -3e^{i\frac{2\pi 2}{4}}+(3+2i)e^{i\frac{2\pi 3}{4}})=3,
\]
\[
f_2=\cdots=-1,\qquad f_3=\cdots=1.
\]}
\end{frame}
%------------------------------------------------------
%------------------------------------------------------
\begin{frame}[fragile]
{Inverz  Diszkrét Fourier transzformáció}
\begin{exampleblock}{1. feladat}
Határozzuk meg a következő vektorok Inverz Diszkrét Fourier transzformáltját! Írjuk fel a megoldást mátrixos alakban is!
\begin{itemize}
\item $x = [2, 3, -1, 1]$;
\item $x = [5, 3-2i, -3, 3+2i]$.
\end{itemize}
\end{exampleblock}
\begin{exampleblock}{2. feladat}
Kérjük le Matlabban az \texttt{fft} és az \texttt{ifft} parancsok helpjét. Ellenőrizzük a korábbi megoldásaink helyességét Matlabbal!
\end{exampleblock}
\end{frame}
%------------------------------------------------------
%------------------------------------------------------
%------------------------------------------------------
%------------------------------------------------------

\end{document}



\begin{frame}[fragile]
{Fourier transzformáció}
\begin{exampleblock}{Példa}
Számítsuk ki Matlab segítségével az $f(x)=e^{-|x|}$ Fourier transzformáltját, majd ábrázoljuk a függvényt és a transzformáltat egy ábrán.
\end{exampleblock}
\textbf{Megoldás:}
A \texttt{fourier} parancs segítségével számítsuk ki a Fourier transzformáltat!
\begin{verbatim}
>> syms t x
>> f = exp(-abs(t));
>> F = fourier(f,t,x)
\end{verbatim}
Ábrázoljuk a függvényt és a transzformáltat közös ábrán 
közös ábrán!
\begin{verbatim}
>> fplot([f F])
>> legend('f','F');
>> axis([-4,4,-.5,2.5])
>> ax = gca;
>> ax.XAxisLocation = 'origin';
>> ax.YAxisLocation = 'origin';
\end{verbatim}
\end{frame}
